\documentclass[a4j,fleqn,10pt]{jsarticle}
\usepackage{ipsj}
\usepackage{txfonts}
\usepackage[symbol*,perpage]{footmisc}
\usepackage[dvipdfm]{graphicx, color}

\begin{document}

\section{要約}
ステガノグラフィ技術の主な目的はステガナリシス技術に対しステゴデータの認知性を最小化しつつ埋め込
み率を最大化することである。しかし、しかしながら他の情報ハイディング技術
とは対照的に、埋め込み者は認知性が採用となるカバーデータを選択する上での自
由がある。この考えは提唱されてきた埋め込み技術の中で全く言及されないまま
だった。本論文では私達は3つのシナリオにそってその問題についての調査を行っ
た。シナリオは埋め込み者がステガノグラフィ技術に対し

\begin{enumerate}
 \item 全くの知識がない
 \item ある程度の知識がある
 \item 完全な知識がある
\end{enumerate}

である。

例えば、2のケースにおいて、私達は研究を等してどんなシンプルな統計的尺度
で埋め込み者を支援することができるのかを例示する。

\section{はじめに}
ほかのデジタルコンテンツとは対照的に、過去数年間のうち、画像の可能性を広
げ、画像の特性に対するよりよい理解得る助けとなるを画像ベースのステガノグ
ラフィに関する研究がたくさんあった。

またそのような技術の発展は文字通り、
たくさんのステガナリシス技術の提案へとつながった。その読者はステガノグラ
フィとステガナリシス技術に対し包括的レビューを論文[1]にて言及している。

現代的なステガノグラフィはシモンズの囚人(アリスとボブという2人の受刑者)
問題[2]において定式化されており、監視員であるウェンディの監視の中、
受刑者は平面図の風景を企てるためにメッセージMを伝えあった。

ステガノグラフィの目的はアリスが秘密のメッセージMをカバーイメージに埋め込むことで
(つまりはステガノグラフィだ)形作られていたのだ。(意訳)

そして、普通なら誰かがウェンディ監視官にもっとばれないようなよりよい埋め
込み技術について証明するだろう。

論文[3]で提唱されているよなベンチマークに基づいたものによって埋め込み技
術が他と比べよいと推定される。

しかし、ステガノグラフィは透かしなどの技術とは違い、カバーデータはメッセー
ジを運ぶ役割しか持たない。埋め込みの家庭でアリスが自由に画像を選んだいた
ように。

これ(Fig.1)はカバーデータの選択における一般的なブロック図だ。ステゴデー
タを手に入れたあと、アリスはステゴデータを送るか代わりの画像を選択するか
どうかを決めるためにステゴデータとカバーデータを比べる。

こうすることで、彼女はステガナライザーによって間違った分類をされる(つま
りは埋め込みがばれない)ようなカバーデータを選択する。

そのため強力なステガナライザーの存在下でさえ、彼女は認知性を下げるチャンスを増やすこと
ができたのだ。

私達は2章で問題提起と解決法を述べ、3章でカバーデータとステゴデータの比較
に使用される尺度について述べる。

私達は4章の実験を通して埋め込み者のカバーデータ選択においてどのような指
標について議論スべきかを解説する。

そして5章にて結果についての考察と今後について述べる。

\section{問題提起}
上述したようにアリスはワーデンがご分類してしまうようなカバーデータを選択
することで、(埋め込みに)気づかれる可能性を最小化することができた。

そのためには彼女はステガナライザーのふるまいを理解する必要があり、そして
(分類)の決定方法に参考にする必要があった。

ステガナライザーの分類機能は画像内の特徴空間におけるハイパープレーンの決
定関する位置に基づいてカバーデータとステゴデータを区別する。

特徴空間における識別性は、埋め込み処理によるステゴデータの統計計算誤差や
カバーデータにおけるそのようなズレによる欠損に依存する。

事実アリスは埋め込み者であり、カバーデータとステゴデータの両方に対して
(ズレを起こすことが?)可能であった。

埋め込み者が埋め込み用に画像をランダムに選択するのではなく選択可能であり
かつ使用可能な画像を多く所有していると仮定し、私達はカバーデータとステゴ
データの関係の評価尺度をベースとした単純ながらも効果的な
(画像選択における)ランク付けの手法を提案します。それによってアリスのス
テゴデータの選択を支援し、より偽陰性を生じさせることができます。

さらに、私達はステゴデータの検出性に影響を与えうる埋め込み法とは別に、カ
バー画像特性(圧縮率、サイズ)についても議論する。

一つに、アリスがステガナライザーに対するさまざまなレベルの知識を要すると
仮定した3つのしなりをを考えることができる。

\begin{enumerate}
 \item 全く知らない:
       彼女はステガナライザーに対する知識を全く有していません。しかし、
       彼女は埋め込みによって変化するであろう一般的な特徴について着目し
       認知を回避する機会を増やすことができます。
       例えば、埋め込み処理に起因するjpegのDCT係数を最小化する。

 \item 多少の知識はある:
       ありすがステガナライザーについて限られた範囲での知識を有していた
       場合。アリスはワーデンのステガナライザーに対していくつかのアクセ
       ス方法を有していたと仮定できる。しかし、彼女のアクセスは2つの方法
       に限られている。
       第一に、彼女はステガナライザーへの入力と出力へのアクセス権のみを
       有している。このように、彼女は内部での働き(特徴的な点)を具体的
       に知ることはできません。
       第二に、彼女のアクセスは彼女が選択した入力画像に対するステガナラ
       イザーに対する決定を入手することが許されているにもかかわらず、時
       間的に制限されている場合である。
       ここでナニも知らない場合と比較すると、彼女はカバーデータの選択に使用さ
       れる特定のしきい値を計算するためのテキストとステゴデータのセット
       と対応する捨て仮名ライザーの決定を使用することでより正確になり得
       る。

 \item よく知っている:
       この場合、アリスはステガナライザーとワーデンによって使用される統
       計的特徴に関する確かな知識を有している。そこで彼女は同様のステガ
       ナライザーを仮定する(※ 直訳だと訓練)ためにカバーデータ
       とステゴデータのデータのセットを使用することができる。
       そして、アリスは仮定したステガナライザーに対してステゴデータのテ
       ストを行うことができ、またご分類されるカバーデータを識別すること
       ができる。この場合私達のアプローチはあまり意味をなさない。
\end{enumerate}

この時点で、ステガナライザーによって私達のアプローチに誤りが生じている(偽陰性によって)可能性あ
ります。私達のランキング付けの手法はその誤り率に依存していることに注意し
てください。

もしステガナライザーの誤りがないのであれば、すべてのステゴデータは検出さ
れるでしょう。そして私達のアプローチは役に立たないでしょう。

次のセクションではいくつものカバーデータのプロパティについてと同様にカバー
データの選択において使用される指標をもとにカバーとステゴの関係についても
調査する。

\section{指標}
\subsection{カバーベース}
埋め込み動作とは独立して使用されるカバーの特性はステガナライザーの性能に
影響を与える。以下にレニューする2つの特性を述べる
\begin{enumerate}
 \item 可変係数
       埋め込み処理によって利用される係数のセット。メッセージの内容は固
       定されているため、たくさんの可変係数を有する画像は比較的埋め込み
       処理によって生じる変化の数が少ない。
 \item JPEGの質の要因
       [3]の論文見てもらったらわかるけどいままでベンチマークとってたんだ
       よね、そりゃもう厳しい実験で扱ってきたさ。
       JPEGの室の要因となる値はステガナライザーの性能と反比例するんだ。
       言い換えるとより高いクオリティのJPEGならより低いパフォーマンスに
       なるってことだね(意訳)
\end{enumerate}

\subsection{カバーとステゴの関係♡ ベース}
まあどっちかしか手に入らないってわけじゃないし、じゃあ両方使って測定しちゃ
おうってわけwだから僕らって興味あるんだよね、生成されたファイルを測定す
る指標ってやつにさ。以下で僕らが採用し、選択の指標となる評価指標について
紹介するよ(意訳)

\begin{enumerate}
 \item カバーデータからの変化量
       単純にカバーとステゴの違いってわけ。超直感的だしわかりやすいよね。
 \item MSE(PSNR)
       (略)
 \item 予測誤差
       論文[4]で提案されてる予測モデルで、僕らの実験だけで使用したまあロー
       カルな指標さ(たぶんグローバルなものではなく自分たちの予測のもと
       に作成したもの)(正直ココはナニが言いたいのかわからん)MSEとの相
       関が予想されてるよ。
 \item ワトソンのメートル法(論文[5])はJPEGの画質の適量化に使用される近
       く的尺度のこと。つまり検出可能性はカバーとステゴの間に生じるこの
       手法の値の差が低い程低いはずである。
 \item SSIM
       (略)
\end{enumerate}

\section{実験}
論文[3]で得られたグレースケール画像の大規模なデータセットから、私達は85
以上の(JPEGの)品質係数を備え、かつ少なくとも1000pxを備える画像を手いにれました。
これらの画像は680☓ 480pxに縮小し、95および75の品質係数でJPEG化しました。
したがって異なる圧縮率を持つ2つの画像セットを約13000の画像から得ました。
ステゴデータは論文[7]の埋め込み法を用いて画像の画素数を元にした固定長の
メッセージとともに生成されます。
2つのステガナライザーにはFBSとWBSを採用しました。(論文[8]と論文[9])

ステガナライザーは(所有するうちの)30%のカバーとステゴによって学習(訓
練)された。
メッセージ長を1ピクセルあたり0.04として品質係数95のJPEG画像と使用した。
75のものでは1ピクセルあたり0.05とした。(論文[3]からも分かる通り、データセットは埋め込み法に依存
する(つまりは埋め込み法によって生成されるステゴが異なるということ)こと
から2つのデータに対する埋め込み率を変更することとした。)

\begin{enumerate}
 \item 全く知らない:
       埋め込み者がステガナライザーがどのようにカバーとステゴを区別する
       のか知らない場合、彼女はステガナライザーの決定と彼女の直感がよい
       相関を持つことを願って、直感によって埋め込みの影響が少なくなる画
       像を選ばなければならない。
       前章ではステガナライザーの出力と相関するであろういくつかの要素に
       ついて述べた。(PSNR...SSIM...)
       カバー選択におけるそれら(PSNR...)の信頼性をテストするため
       に、私達はステガナライザーを設計する上で使用する指標とに学習中にステガナ
       ライザーからは見えない画像(残りの70%?)ついて議論することで研
       究を行った。

       以下に実験手順を示す

       \begin{enumerate}
        \item 全体の10%の画像をランダムに選択し、TPの数を計算する(TP
              is 何???)
        \item 指標に基づいて画像をソートする
        \item 最大もしくは採用の1割の画像を選択し、TP率を計算する。
              (PSNRなら最大、誤り率なら最小といった意味)
       \end{enumerate}

       私達はステガナライザーは約10%以上誤差を孕んでいると仮定して10%の
       最大or最小の10%の画像を選択した。

       もし選択の範囲を少なくすればサンプル数が減ってしまうし、逆に増や
       せば、分類者の誤差率が少なかった場合にステガナライザーによる検出
       の可能性を上げることとなってしまうだろう。

       ステップ1で、私達は10%の画像がランダムに選択されたものだと仮定し
       てステゴデータとして正しく分類されてしまうであろうたくさんの画像
       に対し計算処理を行った。
       
       ステップ3では、私達の単純なランク付け方式に則って正しく識別された
       ステゴデータに対し計算処理を行った。

       結果、2つの埋め込み解析(FBSとWBS)の結果が得られた、表1に示す。

       表1の題名:知識のない中でのよいカバーデータの選択の改良
       補足:(*)がついているものは昇順にソートされたのものです。このような(知識がない)場合、大きい値ほどステゴデータの検出性が低くなった。その他のものは降順ソートしたものです。

       表1より2つ(FBSとWBS)の実験結果において分類結果と指標との相関が
       が異なることがわかる。例えば可変係数の最大値に対してFBSとの識別率
       は正確ではなく、一方でWBSの性能には影響を与えていない。

 \item 多少は知っている
       私達が実験を行った2つ目のシナリオは部分的な知識を備えている場合だ。
       この場合、私達は埋め込み者はいくつかのステゴに対するステガナライ
       ザーの決定を手に入れていると仮定する。
       そのため、アリスは先ほど示した指標についてのしきい値を知るための
       訓練のためにこれらの画像を使用することができる。
       例えば、テストデータのセットから計算された埋め込み過程の一部とし
       て生じたいくつかの変化の数よりも少ない変更をもつ画像を使用すれば
       ステガナライザーに対して認知されにくい。
       
       前章と同じデータセットを使い、私達のアプローチが正確であることを
       調査するために以下のステップを実行した。

       \begin{enumerate}
        \item トレーニング
              \begin{enumerate}
               \item ステガナライザーの決定から得られた画像を使用して、
                     上位または下位1割の画像を選択する。
               \item 選択した画像から得られた指標の平均値を計算する
              \end{enumerate}
       \end{enumerate}

 \item テストする
       トレーニング段階で使用されなかった画像を使って、指標のしきい値で
       分類し(たぶんthresholdで基準をもとにわけるとかってニュアンスに
       なるのかと)(埋め込みの)結果のセットを使ってTPの数を計算する。
       
       表2の題名:FBSの行動予測における部分的な知識における画像選択の改
       善
       補足:表1と同じ

       上記の実験を10回の繰り返し、私達はFBSとWBSそれぞれの組み合わせに
       対しての非検知性の改善の平均値を計算した。表2はFBS組み合わせから
       得られた結果を示している。何も知らない場合と部分的な知識がある場
       合に10,100,500の画像を埋め込みトレーニングに使用した場合のものだ。

       例えば、私達何も知らない状態でMSEにもとづいて画像を選択したとすれ
       ば私達は確実に(FBSによって)発見されているだろう。

       しかし10枚の画像を指標としきい値の調整に使用した場合、私達はTP率
       を99%から30%まで改善することができた。表3に示すように同様の結果
       がWBSでも得られた。

       表3の題名:WBSの〜

\end{enumerate}

\section{結果}
以上の研究で、カバーの選択における問題について言及し、シンプルだが効果的
な解決策について調査を行った。

私達の設定では、アリスはウェンディーの仕訳機に対し限られたアクセスしか持っ
ていなかった。しかしいくつかのステゴテストによる仕訳機の決定についての情
報を持つことで、彼女はよりよいカバーの選択を行うことができた。

このために、彼女はステガナライザーの出力と同様のテストデータのクラス分け
を行うための(しきい値と)(ゆがみの)指標を研究した。

私達は研究を通してどれほどの量のシンプルな指標がステガナライザーによる非
認知性の改善につながるかを述べ、それぞれの指標を用いた場合のパフォーマン
スについて調査した。

私達はMSEのような歪みの指標や誤りの予測は埋め込みによる歪みの質を測る上
でよい指標とはならないということを見ることができた。

それはカバー - ステゴ間の歪みの減少はご認識の可能性を上げることにはなら
ないということを示している。

しかし、表1のように逆は正しいです。

この問題はてステガノグラフィーの理論的なフレームワークの開発になると考え
られます。

例えば、FBSを用いた場合私達はMSEが高いほどにステゴデータが誤分類され
ていることがわかる。

(これはFBSはステゴデータを分割し再圧縮することでカバーデータを評価しているという意味なのか
もしれない。カバーデータに注入された増加した歪みはより不正確な評価を生じ
させ、同様に区別機の性能も小さくなる)

何も知らない場合、私達の結果はカバーに生じる変化を減らすことが信頼できる
指標となることを示している。

一方、部分的な知識を持っている場合はその単純さにもかかわらず、MSEとか変
形数の数はより低い検出性のカバーを選択する上での効果的な指標となった。

SSIMやワトソンのメートル法のような知覚的な指標においてこれらの指標の有意
性もまた埋め込みによる歪みは統計的(というよりは知覚的)な方法で定量化す
る必要があることを示している。

加えて、テストデータの量による部分的な知識がもたらす可能性は(テストデー
タの数を)10から100に増やすことでより効果的なものとなり、500に増やすこと
はあまり大きな差を生じさせることはなかった。

私達は現在、代わりの指標を調査するだけでなく、より多くの埋め込み発見法をカバーしよう
と私達の仕事を拡大している。

\end{document}
